\documentclass[b5j]{jarticle}

%%朝プリsetting
\usepackage[driver=dvipdfm,top=1cm,bottom=1.5cm,left=1cm,right=1cm,foot=1cm]{geometry}%用紙設定
\pagestyle{empty}%ページを出力しない%
%\usepackage{EMpict2e}

%\def\thesection{No. \arabic{section}}
%
%\def\Name#1{\section{\large\bf #1\hfill
%\underline{ \hspace{1zw}年\hspace{2zw}組\hspace{2zw}
%番名前\hspace{12zw}}}}

\def\thesection{No. \arabic{section}}
\def\Name#1{\section{\large\bf  #1\hfill
\underline{ \hspace{1zw}年\hspace{2zw}組\hspace{2zw}
番名前\hspace{12zw}}}}

\def\thesubsection{\arabic{subsection}}

\def\labelenumi{(\theenumi)}
\def\theenumii{\roman{enumii}}
\def\labelenumii{(\theenumii)}
%\def\theenumiii{\roman{enumiii}}
%\def\labelenumiii{(\theenumiii )}




%{\LARGE \hfill 極限\hfill No.1  }
%\begin{center}
%
%\hfill{\Large \underline{\hspace{1zw}年\hspace{2zw}組\hspace{2zw}
%番名前\hspace{12zw}}}
%\end{center}


%\usepackage[dvipdfmx,svgnames]{xcolor}%tikzパッケージよりも前に読み込みます。
%
%\usepackage{tikz} % tikzパッケージを読み込む
%\usetikzlibrary{intersections, calc}
%\usetikzlibrary{quotes, angles,through,patterns}%quotesは角度のラベルオプション内に記述するために使用
%\usepackage{tikz-3dplot}
%\usepackage{mathrsfs}
%\usetikzlibrary{arrows}



%\usepackage[deluxe]{otf}%和文フォント
%\usepackage{latexsym}
%\usepackage{amssymb}
%\usepackage{MnSymbol}
%$\largestar$
%\usepackage{amsmath}
%\usepackage{amsthm}
\usepackage{ulem}%下線,波線,取り消し線など
\usepackage{tabularx}%伸びる表

\usepackage{emathMw}


\usepackage{emathP}%emathスタイルファイル
\usepackage{emathEy}%問題番号枝など
\usepackage{emathGps}

%\usepackage[shortlabels,inline]{enumitem}
%箇条書き環境

%\setlist{noitemsep}
%\setlist[enumerate]{leftmargin=*}
%\setlist[enumerate,1]{label=\large\textbf{\arabic*.}, ref=\arabic*}
%\setlist[enumerate,2]{label={(\arabic*)}, ref={(\arabic*)}}
%\setlist[enumerate,3]{label={\roman*)}, ref={\roman*)}}
%
%
%
%\def\labelenumi{\Large\textbf\theenumi. }
%\def\theenumii{\arabic{enumii}}
%\def\labelenumii{(\theenumii)}
%\def\theenumiii{\roman{enumiii}}
%\def\labelenumiii{(\theenumiii )}%問題番号
%

\usepackage{hako}%ハコ環境
\usepackage{multicol}%段落環境
\setlength{\columnsep}{1cm}%段落間隔

%\usepackage[continue]{emathAe}%解答環境
%\usepackage{type1cm}% PS, PDF 作成には必要
%\usepackage[debug]{emathPs}%描画eps作成
\usepackage{showexample}%\itemboxなど
%
%
%%%%%%%ベクトル%%%%%%
\usepackage[e]{EMesvect}%ベクトル
\def\Beku#1{\ooalign{\bekutoru{ }\crcr\hss{$#1$}\hss}}

%%%%%太字%%%%
%\def\bf{\gtfamily\bfseries}
%



\begin{document}


\setcounter{section}{0}



%%51-55 2008/6%%
%% 51 %%
\Name{数学IAIIB応用}
\hakosyokika
\begin{caprm}
  Githubのtestだよう。
  \LaTeX できた。再起動が大切だ。

$OA=4$,$OB=1$,$\kaku{AOB}=60\Deg$である三角形OABにおいて,
$\bekutoru{OA}=\Beku a$,$\bekutoru{OB}=\Beku b$とする。また,辺OAを$s:(1-s)$に内分する点をC,辺ABを$t:(1-t)$に内分する点をDとする。ただし,$0<s<1$,$0<t<1$である。
\begin{enumerate}
\item $\bekutoru{CD}$を$s$,$t$,$\Beku a$,$\Beku b$を用いて表せ。
\item $\bekutoru{CD}\perp\Beku b$のとき,$s$を$t$を用いて表せ。
\item (2)の条件の下で,$\zettaiti{\bekutoru{CD}}=\bunsuu1{\sqrt 3}$が成り立つとき,$t$の値を求めよ。
\end{enumerate}%8
\end{caprm}


\newpage

%% 52 %%
\Name{数学IAIIB応用}
\hakosyokika
次の連立方程式を解け。ただし,$x<y$とする。
$$x^{2}+y^{2}+3xy=11,\,x+y-xy=9$$
%32
\newpage

%% 53 %%
\Name{数学IAIIB応用}
\hakosyokika
\begin{enumerate}
\item $\zettaiti{x^{2}-1}=x+1$を解け。
\item $\zettaiti{x^{2}-1}=x+a$が3個の解をもつとき,$a$の値を求めよ。
\end{enumerate}%33
%\begin{enumerate}
%\item $\dlim{x\to -2}\bunsuu{1}{x+2}\left(\bunsuu{1}{x^{2}+1}+\bunsuu1{x-3}\right)=\Hako$

%\item $\dlim{n\to \infty}\bunsuu{1+3+5\cdots+(2n-1)}{2n^{2}+3n-1}=\Hako$

%\item $\dlim{x\to -\infty}\bunsuu{\sqrt{x^{2}+x+1}}{x}=\Hako$

%\item $\dlim{x\to \infty}\left\{\sqrt{4x^{2}+3x}-2x\right\}=\Hako$

%\end{enumerate}%42
\newpage

%% 54 %%
\Name{数学IAIIB応用}
\hakosyokika
9枚のカードがあり,各々のカードには$1,\,\sqrt2,\,\sqrt3,\,2,\,\sqrt5,\,\sqrt6,\,\sqrt7,\,\sqrt8,\,3$が書かれている。これらのカードを横一列に並べるとき,どの無理数のカードの右側にも,その数よりも大きい整数のカードがあるような確率を求めよ。
%68
\newpage

%% 55 %%
\Name{数学IAIIB応用}
\hakosyokika
0以上の実数$s$,$t$が$s^{2}+t^{2}=1$を満たしながら動くとき,方程式$x^{4}-2(s+t)x^{2}+(s-t)^{2}=0$の解のとる値の範囲を求めよ。
%57
\newpage


%%56-60 2006/7
%% 56 %%
\Name{数学IAIIB応用}
\hakosyokika
\begin{caprm}
直線$l:y=\bunsuu12x+1$と2点$A(1,\,4)$,$B(5,\,6)$がある。
\begin{enumerate}
\item 直線$l$に関して,点Aと対称な点Cの座標を求めよ。
\item 直線$l$上の点Pで,$AP+PB$を最小にするものの座標を求めよ。
\end{enumerate}
\end{caprm}%8
\newpage

%% 57 %%
\Name{数学IAIIB応用}
\hakosyokika
\begin{caprm}
座標平面上の2点Q$(1,\,1)$,R$\left(2,\,\bunsuu12\right)$に対して,点Pが円$x^{2}+y^{2}=1$の周上を動くとき,
\begin{enumerate}
\item \sankaku{PQR}の重心Gの軌跡を求めよ。
\item 点Pから\sankaku{PQR}の重心Gまでの距離が最小となるとき,点Pの座標を求めよ。

\item \sankaku{PQR}の面積の最小値を求めよ。

\end{enumerate}

\end{caprm}
%14
\newpage

%% 58 %%
\Name{数学IAIIB応用}
\hakosyokika
正の数$a$,$b$,$x$,$y$を考える。$a+b=1$ならば,すべての自然数$n$について不等式$(ax+by)^{n}\leqq ax^{n}+by^{n}$であることを証明せよ。
%37
\newpage

%% 59 %%
\Name{数学IAIIB応用}
\hakosyokika
\begin{caprm}
放物線${\it C}:y=\bunsuu14x^{2}$と,点B$(0,\,b)$を中心とする半径$r$の円(ただし,$r<b$)は異なる2点$A_{1}$,$A_{2}$を共有し,それ以外には共有点をもたないとする。ここで,点$A_{1}$,$A_{2}$はそれぞれ,第1象限,第2象限にあるとする。

\begin{enumerate}
\item 円の中心Bおよび2点$A_{1}$,$A_{2}$の座標を
$r$を用いて表せ。さらに,$r$の値の範囲を求めよ。

\item $r=4$のとき,$\kaku{}A_{1}BA_{2}$の大きさを求めよ。ただし,$0\Deg<\kaku{}A_{1}BA_{2}<180\Deg$とする。



\item (2)で求めた$\kaku{}A_{1}BA_{2}$に対する弧$\ko*{A_{1}A_{2}}$と放物線${\it C}$で囲まれた図形の面積を求めよ。

\end{enumerate}


\end{caprm}%15

\newpage

%% 60 %%
\Name{数学IAIIB応用}
\hakosyokika
任意の実数$x$に対して常に$ax^{2}+2bx+c>px^{2}+2qx+r>0\cdots☆$
が成立することき,$ac-b^{2}>pr-q^{2}$が成立することを示せ。ただし,$a\neqq p$,$ap\neqq 0$とする。
%59

\newpage



%%61-65 2008/8%%
%% 61 %%
\Name{数学IAIIB応用}
\hakosyokika
数列$\{a_{n}\}$が$a_{1}=2$,$(n+2)a_{n+1}-na_{n}=3n+4\quad (n\geqq 1)$で定義されるとき,
\begin{enumerate}
\item $a_{2}$,$a_{3}$,$a_{4}$を求めよ。

\item $a_{n}$を$n$の式で表せ。
\end{enumerate}%4

\newpage


%% 62 %%
\Name{数学IAIIB応用}
\hakosyokika
3の倍数でない自然数を小さいものから順に並べたとき,$n$番目の数を$a_{n}$とする。例えば,$a_{1}=1$,$a_{2}=2$,$a_{3}=4$,$a_{4}=5$,$a_{5}=7$,$a_{6}=8$,$a_{7}=10$,$a_{8}=11$,$\cdots$である。
\begin{enumerate}
\item $a_{20}$を求めよ。

\item $\tretuwa{k=1}{20}a_{k}$の値を求めよ。

\item $a_{n}=2008$となる$n$の値を求めよ。
\end{enumerate}
%8
\newpage

%% 63 %%
\Name{数学IAIIB応用}
\hakosyokika

$0\leqq x<\pi$のとき,関数$f(x)=a\cos^{2}x+b\sin^{2}x+\cos x\sin x$について,次の問いに答えなさい。ただし,$a$,$b$は実数である。
\begin{enumerate}
\item $f(x)$を$\cos 2x$と$\sin 2x$を用いて表せ。

\item $f(x)$の最大値と最小値をそれぞれ$a$,$b$を用いて表せ。

\item $f(x)$の最大値と最小値がをそれぞれ$2$,$-1$であるとき,$a$,$b$の値を求めよ。
\end{enumerate}
%36

\newpage



%% 64 %%
\Name{数学IAIIB応用}
\hakosyokika
\begin{caprm}
平面上に\sankaku{ABC}がある。この平面上の点Pに対してAPの中点をQ,BQの中点をR,CRの中点をSとする。2点P,Sが一致しているとき,
\begin{enumerate}
\item $\bekutoru{AB}=\Beku b$,$\bekutoru{AC}=\Beku c$とするとき,$\bekutoru{AP}$を$\Beku b$,$\Beku c$を用いて表せ。

\item \sankaku{PQR}と\sankaku{ABC}の面積比を求めよ。
\end{enumerate}
\end{caprm}
%63

\newpage



%% 65 %%
\Name{数学IAIIB応用}
\hakosyokika
自然数$n$に対して,$a_{n}=(\cos2^{n})(\cos2^{n-1})\cdots(\cos2)(\cos1)$とおく。ただし,角の大きさは弧度法を用いる。
\begin{enumerate}
\item $a_{1}=\bunsuu{\sin4}{4\sin1}$を示せ。
\item $a_{n}=\bunsuu{\sin2^{n+1}}{2^{n+1}\sin1}$を示せ。
\item $a_{n}<\bunsuu{\sqrt 2}{2^{n+1}}$を示せ。
\end{enumerate}
%37

\newpage





%%66-70 2008/9%%
%% 66 %%
\Name{数学IAIIB応用}
\hakosyokika
6人の学生を3組に分ける。まず,3人,2人,1人の3組に分ける方法は\Hako 通りある。次に2人ずつの3組に分ける方法は\Hako 通りある。6人の学生を3組に分ける方法は全部で\Hako 通りある。
%8

\newpage


%% 67 %%
\Name{数学IAIIB応用}
\hakosyokika
1,2,3,4を重複を許して並べてできる数について,
\begin{enumerate}

\item 各桁の数の和が6となる5桁の数の個数を求めよ。

\item 各桁の数の和が7となる5桁の数の個数を求めよ。

\item 各桁の数の和が$k+3$となる$k$桁の数の個数を求めよ。



\end{enumerate}

%8
\newpage

%% 68 %%
\Name{数学IAIIB応用}
\hakosyokika

白い玉が2個,黒い玉が3個,赤い玉が4個ある。これらの玉を次のような条件ですべて使って,一列に並べる。
\begin{enumerate}
\item 玉の色のすべての並べ方は,\Hako 通りである。
\item 白い玉2個が隣り合わない並べ方は,\Hako 通りある。
\item 黒い玉3個が連続している並べ方は,\Hako 通りある。
\item 同じ色な玉は連続しない並べ方は,\Hako 通りある。


\end{enumerate}
%14

\newpage



%% 69 %%
\Name{数学IAIIB応用}
\hakosyokika
6人がそれぞれプレゼントを持参してパーティーに参加した。参加者が自分以外の誰かにプレゼントを渡すとき,6人全員が1つずつプレゼントを受け取ることができるような渡し方は\karaHako 通りある。
%14

\newpage



%% 70 %%
\Name{数学IAIIB応用}
\hakosyokika
空間に座標系が定められていて,$z$軸上に2点A$(0,\,0,\,6)$,B$(0,\,0,\,20)$が与えられている。$xy$平面上の点P$(x,\,y,\,0)$で,$0\leqq x\leqq 15$,$0\leqq y\leqq 15$,$\kaku{APB}\geqq30\Deg$を満たすものの全体が作る図形の面積を求めよ。
%40

\newpage





\setcounter{section}{0}



%%51-55 2008/6%%
%% 51 %%
\Name{数学IAIIB応用}
\hakosyokika
\begin{caprm}
$OA=4$,$OB=1$,$\kaku{AOB}=60\Deg$である三角形OABにおいて,
$\bekutoru{OA}=\Beku a$,$\bekutoru{OB}=\Beku b$とする。また,辺OAを$s:(1-s)$に内分する点をC,辺ABを$t:(1-t)$に内分する点をDとする。ただし,$0<s<1$,$0<t<1$である。
\begin{enumerate}
\item $\bekutoru{CD}$を$s$,$t$,$\Beku a$,$\Beku b$を用いて表せ。
\item $\bekutoru{CD}\perp\Beku b$のとき,$s$を$t$を用いて表せ。
\item (2)の条件の下で,$\zettaiti{\bekutoru{CD}}=\bunsuu1{\sqrt 3}$が成り立つとき,$t$の値を求めよ。
\end{enumerate}%8




\begin{multicols*}{2}
{\bf 【解答】}
\begin{enumerate}
\item $\Beku c=\bekutoru{OC}$,$\Beku d=\bekutoru{OD}$とおくと,
$$\Beku c=s\Beku a,\,\Beku d=(1-t)\Beku a+t\Beku b$$
\begin{align*}
\bm{\bekutoru{CD}}&=\Beku d-\Beku c\\
&=\bm{(1-t-s)\Beku a+t\Beku b}
\end{align*}

\item $\Beku a\cdot\Beku b=1\cdot4\cdot\cos60\Deg =2\cdots\maru1$
\begin{align*}
\bekutoru{CD}\cdot\Beku b&=(1-t-s)\Beku a\cdot \Beku b+t\zettaiti{\Beku b}^{2}\\
&=2(1-t-s)+t\quad (\because\maru1,\,OB=1)\\
&=2-t-2s=0
\end{align*}
$$\therefore \bm{s=1-\bunsuu t2}$$

\item
$\begin{aligned}[t]
\zettaiti{\bekutoru{CD}}^{2}&=(1-t-s)^{2}\zettaiti{\Beku a}^{2}+2t(1-t-s)\Beku a\cdot \Beku b+t^{2}\zettaiti{\Beku b}^{2}\\
&=\left(-\bunsuu t2\right)^{2}\times 16+2t\left(-\bunsuu t2\right)\cdot 2+t^{2}\\
&\qquad\quad (\because\maru1,\,OA=4,\,OB=2)\\
&=3t^{2}=\bunsuu13
\end{aligned}$
$\therefore t^{2}=\bunsuu19$。$0<t<1$より,
$$\bm{t=\bunsuu13}$$
\end{enumerate}


{\bf 【コメント】}

図形的にも気になるところですが,今回は代数的に解いただけです。
点CとDの位置に気をつけましょう。
\end{multicols*}


\end{caprm}


\newpage

%% 52 %%
\Name{数学IAIIB応用}
\hakosyokika
次の連立方程式を解け。ただし,$x<y$とする。
$$x^{2}+y^{2}+3xy=11,\,x+y-xy=9$$
%32



\begin{multicols*}{2}
{\bf 【解答】}

$u=x+y$,$v=xy$とおくと,与えられた不等式は
$$(x+y)^{2}+xy=11,\,x+y-xy=9$$
$$\therefore u^{2}+v=11\cdots\maru1,\,u+v=9\cdots\maru2$$
$v$を消去して,
$$u^{2}+u-20=0$$
$$\therefore (u+5)(u-4)=0$$
$$\therefore u=-5,\,4$$

\begin{enumerate}[(i)]
\item $u=-5$のとき,\maru2より,$v=-14$

このとき,$x$,$y$は2次方程式$t^{2}+5t-14=0$の2解となる。
$$(t-2)(t+7)=0$$
$x<y$より,$(x,\,y)=(-7,\,2)$

\item $u=4$のとき,\maru2より,$v=-5$

このとき,$x$,$y$は2次方程式$t^{2}-4t+5=0$の2解となる。
$$(t-5)(t+1)=0$$
$x<y$より,$(x,\,y)=(-1,\,5)$


\end{enumerate}
以上より,$\bm{(x,\,y)=(-7,\,2),\,(-1,\,5)}$


{\bf 【コメント】}

2式とも,$x$,$y$についての対称式なので,$x+y=u$,$xy=v$とおいて考えます。

1文字消去で$y$を消去すると,$x$の4次方程式になります。
$$x^4+x^3-39x^2+31x+70=0$$
因数分解して,
$$(x+1)(x-2)(x+7)(x-5)=0$$
より,解いていきます。
\end{multicols*}



\newpage

%% 53 %%
\Name{数学IAIIB応用}
\hakosyokika
\begin{enumerate}
\item $\zettaiti{x^{2}-1}=x+1$を解け。
\item $\zettaiti{x^{2}-1}=x+a$が3個の解をもつとき,$a$の値を求めよ。
\end{enumerate}%33
%\begin{enumerate}
%\item $\dlim{x\to -2}\bunsuu{1}{x+2}\left(\bunsuu{1}{x^{2}+1}+\bunsuu1{x-3}\right)=\Hako$

%\item $\dlim{n\to \infty}\bunsuu{1+3+5\cdots+(2n-1)}{2n^{2}+3n-1}=\Hako$

%\item $\dlim{x\to -\infty}\bunsuu{\sqrt{x^{2}+x+1}}{x}=\Hako$

%\item $\dlim{x\to \infty}\left\{\sqrt{4x^{2}+3x}-2x\right\}=\Hako$

%\end{enumerate}%42




\begin{multicols*}{2}
{\bf 【解答】}

\begin{enumerate}
\item
\begin{enumerate}[(i)]
\item $x\leqq -1,\,1\leqq x$のとき,
$$x^{2}-1=x+1$$
$$x^{2}-x-2=0$$
$$(x-2)(x+1)=0$$
$$\therefore x=2,\-1(ともに適)$$

\item $-1<x<1$のとき,
$$-x^{2}+1=x+1$$
$$x^{2}+x=0$$
$$x(x+1)=0$$
$$\therefore x=0(x=-1は不適)$$
\end{enumerate}
以上より,$\bm{x=-1,\,0,\,2}$

\item $y=\zettaiti{x^{2}-1}$と$y=x+a$のグラフを考える。
\begin{enumerate}[(i)]
\item $x\leqq -1,\,1\leqq x$のとき,
$$x^{2}-1=x+a$$
$$x^{2}-x-1-a=0$$
$$\left(x-\bunsuu12\right)^{2}-\bunsuu54-a=0$$
2つのグラフの接点はない。
\item $-1<x<1$のとき,
$$-x^{2}+1=x+a$$
$$x^{2}+x-1+a=0$$
$$\left(x+\bunsuu12\right)^{2}-\bunsuu54+a=0$$
よって,$a=\bunsuu54$のとき,$x=-\bunsuu12$で2つのグラフは接する。

\begin{center}
\begin{zahyou}[ul=10mm,gentenhaiti={[se]}](-3,3)(-2,5)
 \def\Fx{abs(X**2-1)}
 \def\Gx{X-1}
 \def\Hx{X+1}
 \def\Ix{X+5/4}

  \YGurafu*\Fx
  \YGurafu*\Gx
  \YGurafu*\Hx
  \YGurafu*\Ix



 \Put{(-1,0)}[s]{$-1$}
 \Put{(-.5,.75)}[syaei=x,xlabel=-\frac12]{}
 \Put{(0,1)}[e]{$1$}
 \Put{(1,0)}[s]{$1$}
 \Put{(0,-1)}[e]{$-1$}
\Put{(0,1.25)}[nw]{$\frac54$}
 \Put{(2,3)}[syaei=x]{}


\end{zahyou}
 \end{center}

共有点の個数が3となるのは,
$$\bm{a=\bunsuu54,\,1}$$
\end{enumerate}




\end{enumerate}



{\bf 【コメント】}

$a$の範囲を共有点の個数で場合分けすると次のようになります。
\begin{align*}
a>\bunsuu54&で2個\\
a=\bunsuu54&で3個\\
1<a<\bunsuu54&で4個\\
a=1&で3個\\
-1<a<1&で2個\\
a=-1&で1個\\
a<-1&で0個
\end{align*}
\end{multicols*}


\newpage

%% 54 %%
\Name{数学IAIIB応用}
\hakosyokika
9枚のカードがあり,各々のカードには$1,\,\sqrt2,\,\sqrt3,\,2,\,\sqrt5,\,\sqrt6,\,\sqrt7,\,\sqrt8,\,3$が書かれている。これらのカードを横一列に並べるとき,どの無理数のカードの右側にも,その数よりも大きい整数のカードがあるような確率を求めよ。
%68


\begin{multicols*}{2}
{\bf 【解答】}

並べ方の総数は$9\kaizyou(通り)$

題意を満たす並べ方を考える。

\begin{description}
\item[手順1]3を右端に配置(1通り)
\item[手順2]3の左側に$\sqrt 5,\,\sqrt 6,\,\sqrt 7,\,\sqrt 8$を配置($4\kaizyou$通り)
\item[手順3]2を【手順2】で並べた各数の間,または両端に配置(6通り)
\item[手順4]$\sqrt 2,\,\sqrt 3$を【手順3】で並べた各数の間,または左端に配置。(右端は2か3なので,さらに右側に$\sqrt 2,\,\sqrt 3$を置くと題意を満たさない。また,$\sqrt 2,\,\sqrt 3$は連続して並んでもよい。)($\bunsuu{7\kaizyou}{5\kaizyou}=7\cdot 6$通り)
\item[手順5]1を【手順4】で並べた各数の間,または両端に配置(9通り)
\end{description}

以上より,$1\times 4\kaizyou\times 6\times7\times 6\times 9$通りとなり,求める確率は,
 $$\bunsuu{1\times 4\kaizyou\times 6\times7\times 6\times 9}{9\kaizyou}=\bm{\bunsuu{3}{20}}$$
{\bf 【コメント】}

日本語が難しいね。『どの無理数のカードの右側』とは右隣だけではありません。もし右隣だけをさすのであれば,並べられないので確率は$0$です。このように,問題文に都合よく解釈をする場合があります。

また,『その数よりも大きい整数のカードがある』の『ある』は存在を表します。つまり,一つあればよいのです。\\

正解者の数え方で多かったのは,3の位置で場合分けしたものです。

{\bf 【別解】}
\begin{enumerate}[m]
\item 3が1番右端$\cdots$
残りはどのように並べてもよい。
$$8\kaizyou 通り$$

\item 3が右から2番目$\cdots$
1番右端は1または2。残りはどのように並べてもよい。
$$2\times 7\kaizyou 通り$$
\item 3が右から3番目$\cdots$
3の右側は
$$(1,2)(2,1)(\sqrt 2,2)(\sqrt 3,2)$$
の4通り。残りはどのように並べてもよい。
$$4\times 6\kaizyou 通り$$

\item 3が右から4番目$\cdots$
3の右側は
$$(1,\sqrt 2,2)(\sqrt 2,1,2)(\sqrt 2,2,1)(1,\sqrt 3,2)$$
$$(\sqrt 3,1,2)(\sqrt 3,2,1)(\sqrt 2,\sqrt 3,2)(\sqrt 3,\sqrt 2,2)$$
の8通り。残りはどのように並べてもよい。
$$8\times 5\kaizyou 通り$$


\item 3が右から5番目$\cdots$
3の右側は$1,2,\sqrt 2,\sqrt3$が並ぶが,
1はどこでもよい。(4通り)\\
そのそれぞれに対して,$2,\sqrt 2,\sqrt3$の並びは
$$(\sqrt 2,\sqrt 3,2)(\sqrt 3,\sqrt 2,2)$$
の2通り。よって,$4\times 2=8$通り。
残りはどのように並べてもよい。
$$8\times 4\kaizyou 通り$$


\end{enumerate}
\begin{align*}
\therefore \quad &\bunsuu{8\kaizyou+2\times7\kaizyou+4\times6\kaizyou+8\times 5\kaizyou+8\times 4\kaizyou}{9\kaizyou}\\
&=\bunsuu{8\cdot7\cdot6\cdot5+2\cdot7\cdot6\cdot5+4\cdot6\cdot5+8\cdot5+8}{9\cdot8\cdot7\cdot6\cdot5}\\
&=\bunsuu{(56+14+4)\cdot30+8\cdot 6}{9\cdot8\cdot7\cdot6\cdot5}\\
&=\bunsuu{37\cdot5+4}{9\cdot4\cdot7\cdot5}=\bunsuu{189}{9\cdot4\cdot7\cdot5}=\bm{\bunsuu{3}{20}}
\end{align*}


\end{multicols*}

\newpage

%% 55 %%
\Name{数学IAIIB応用}
\hakosyokika
0以上の実数$s$,$t$が$s^{2}+t^{2}=1$を満たしながら動くとき,方程式$x^{4}-2(s+t)x^{2}+(s-t)^{2}=0$の解のとる値の範囲を求めよ。
%57


\begin{multicols*}{2}
{\bf 【解答】}

$u=s+t$,$v=st$とおく。

$s$,$t$は$s^2+t^2=1$を満たすので,
$$(s+t)^2-2st=1$$
$$\therefore u^2-2v=1\quad \therefore v=\bunsuu{u^2-1}2\cdots\maru1$$
$s$,$t$は0以上の実数より,$z^2-uz+v=0$がの解が共に0以上と考えて,
$$D=u^2-4v\geqq 0\wedge u\geqq0\wedge v\geqq0\cdots\maru2$$
\maru1かつ\maru2より
$$v=\bunsuu{u^2-1}2\quad(1\leqq u\leqq \sqrt 2)\cdots☆$$

$x^{4}-2(s+t)x^{2}+(s-t)^{2}=0$は
$$x^{4}-2ux^{2}+u^2-4v=0$$
$$\therefore v=\bunsuu{(u-x^2)^2}4\cdots★$$


★のグラフが$\left(\sqrt 2,\,\bunsuu12\right)$を通るとき,
$$\bunsuu12=\bunsuu{\left(\sqrt2-x^2\right)^2}4$$
$$\left(\sqrt2-x^2\right)^2=2$$
$$\sqrt2-x^2=\pm\sqrt2\quad x^2=0,\,2\sqrt 2$$

★のグラフは頂点の座標が$(x^2,\,0)$なので,グラフは$u$軸に接するように動く。
$uv$平面で,☆と★のグラフが共有点を持つような$x$の範囲は

\begin{center}
\begin{zahyou}[ul=10mm,yokozikukigou={$u$},tatezikukigou={$v$}](-2,5)(-1,2)
 \def\Fx{.5*X**2-.5}
 \def\Gx{.25*X**2}
 \def\Hx{.25*(X-sqrt(8))**2}
 \def\Ix{.25*(X-sqrt(2))**2}

 \YTen[xformat=f]\Fx{sqrt(2)}\A
 \YTen[xformat=f]\Fx{-1*sqrt(2)}\B
 \tenretu*{C(1,0)}
 \Put\C[s]{$1$}

 %\YGurafu\Fx{-1*sqrt(2)}{sqrt(2)}
 \YGurafu\Gx{-2}{2}
 \YGurafu\Hx{sqrt(8)-2}{sqrt(8)+2}
 \YGurafu\Ix{sqrt(2)-2}{sqrt(2)+2}
  \YGurafu\Fx\xmin{1}
  \YGurafu\Fx{sqrt(2)}\xmax
 \Put\A[syaei=xy,xlabel=\sqrt 2,ylabel=\frac12]{}
\Put\B[syaei=xy,xlabel=-\sqrt 2,ylabel=]{}
 \thicklines
 \YGurafu\Fx{1}{sqrt(2)}
 \kuromaru{\A;\B;\C}
 \thinlines
\end{zahyou}
 \end{center}

$$0\leqq x^2\leqq 2\sqrt{2}$$
$$\therefore -\sqrt{2\sqrt2}\leqq x\leqq \sqrt{2\sqrt2}$$
$$\therefore \bm{-\sqrt[4]{8}\leqq x\leqq \sqrt[4]{8}}$$

\columnbreak
{\bf 【別解:理系用】}

$x^{4}-2(s+t)x^{2}+(s-t)^{2}=0$を解くと,
\begin{align*}
x^{2}&=s+t\pm\sqrt{(s+t)^{2}-(s-t)^{2}}\\
&=s+t\pm\sqrt{4st}\quad (s,\,tは0以上)\\
&=(\sqrt s\pm\sqrt t)^{2}\\
\therefore x&=\pm\sqrt{s}\pm\sqrt t(複合任意)
\end{align*}

$p=\sqrt s,\,q=\sqrt t$とおくと,
$$p^{4}+q^{4}=1,\,p\geqq 0,\,q\geqq 0$$
\begin{enumerate}[(i)]
\item $x=p+q$のときは,

\begin{multicols}{2}

\begin{center}
\begin{zahyou}[ul=10mm,gentenhaiti={[se]},tatezikukigou={$q$},yokozikukigou={$p$},](-.5,2)(-.5,2)
 \def\Ft{sqrt(cos(T))}
 \def\Gt{sqrt(sin(T))}

 \funcval\Ft{$pi/4}\tx

 \def\Hx{-X+2*(\tx)}
 \def\Ix{-X+1}


  \BGurafu\Ft\Gt{0}{$pi/2}
  \YGurafu*\Hx
  \YGurafu*\Ix
   \Hasen{\O(2,2)}
 \Put{(1,0)}[s]{$1$}
 \Put{(0,1)}[w]{$1$}


\end{zahyou}
 \end{center}
\columnbreak


$(p,\,q)=(1,\,0),\,(0,\,1)$のとき,$x=1$

$p=q=\bunsuu{1}{\sqrt[4]{2}}$のとき,$x=\sqrt[4]{8}$

\end{multicols}

$$\therefore 1\leqq x\leqq\sqrt[4]{8}\cdots\maru1$$

\item $x=-p-q$のときは,同様に,
$ 1\leqq -x\leqq\sqrt[4]{8}$
$$ \therefore -\sqrt[4]{8}\leqq x\leqq-1\cdots\maru2$$


\item $x=-p+q$のときは,

\begin{multicols}{2}
\begin{center}
\begin{zahyou}[ul=10mm,gentenhaiti={[se]},tatezikukigou={$q$},yokozikukigou={$p$},](-.5,2)(-1.5,1.5)
 \def\Ft{sqrt(cos(T))}
 \def\Gt{sqrt(sin(T))}

 \funcval\Ft{$pi/4}\tx

 \def\Hx{X+1}
 \def\Ix{X-1}


  \BGurafu\Ft\Gt{0}{$pi/2}
  \Drawline{(-.5,.5)(.5,1.5)}
  \YGurafu*\Ix

 \Put{(1,0)}[s]{$1$}
 \Put{(0,1)}[w]{$1$}
\Put{(0,-1)}[e]{$-1$}


\end{zahyou}
 \end{center}
\columnbreak


$(p,\,q)=(1,\,0)$のとき,$x=-1$

$(p,\,q)=(0,\,1)$のとき,$x=1$
\end{multicols}
$$\therefore -1\leqq x\leqq1\cdots\maru3$$


\item $x=-p+q$のときは,同様に,
$ -1\leqq -x\leqq1$
$$ \therefore -1\leqq x\leqq1\cdots\maru4$$


\end{enumerate}

\maru1から\maru4の和集合を考えて,
$$\bm{-\sqrt[4]{8}\leqq x\leqq \sqrt[4]{8}}$$

{\bf 【コメント】}

$x^{n}+y^{n}=1$のグラフの概形は既知としました。(文科系はちょっときついか....)
\end{multicols*}

\newpage


%%56-60 2008/7
%% 56 %%
\Name{数学IAIIB応用}
\hakosyokika
\begin{caprm}
直線$l:y=\bunsuu12x+1$と2点$A(1,\,4)$,$B(5,\,6)$がある。
\begin{enumerate}
\item 直線$l$に関して,点Aと対称な点Cの座標を求めよ。
\item 直線$l$上の点Pで,$AP+PB$を最小にするものの座標を求めよ。
\end{enumerate}



\begin{multicols*}{2}
{\bf 【解答】}

\begin{enumerate}
\item Aから直線$l$に下ろした垂線の足をHとする。

Hは直線$l$上なので,媒介変数$t$を用いて,
$$H(2t,\,t+1)$$
とおける。$AH\perp 直線l$,直線$l$の方向ベクトルは$\Beku d=\retube21$より,
\begin{align*}
\bekutoru{AH}\cdot\Beku d&=\retube{2t-1}{t+1-4}\cdot\retube21\\
&=4t-2+t-3\\
&=5t-5=0\\
&\therefore t=1
\end{align*}
$$\therefore H(2,\,2)$$
$C(x,\,y)$とおくと,$\bekutoru{AC}=2\bekutoru{AH}$より,
$$\retube{x-1}{y-4}=2\retube{1}{-2}$$
$$\therefore x=3,\,y=0\quad \therefore \bm{C(3,\,0)}$$

\item $A$,$B$は直線$l$に関して同じ側にある。
\begin{center}
\begin{zahyou}[ul=10mm,gentenhaiti={[se]},tatezikukigou={$y$},yokozikukigou={$x$},](-.5,6)(-.5,7)
 \def\Fx{.5*X+1}
 \tenretu{A(1,4)n;B(5,6)ne;C(3,0)s;P(2.5,2.25)se;H(2,2)sw}
 \YGurafu*\Fx

  \Hasen{\A\C\P}
  \Hasen{\C\B}
  \Drawline{\A\P\B}
  \kuromaru{\A;\B;\C;\P}
\Tyokkakukigou\A\H\P
\touhenkigou<2>{\A\H;\H\C}



\end{zahyou}
 \end{center}

よって,$AP=CP$より,
$$AP+PB=CP+PB\geqq CB(一定)$$
直線CBの式は
$$y=3(x-3)$$
よって,直線$l$とCBの交点がPとなるので,連立して解くと,
$$3(x-3)=\bunsuu12x+1\quad \therefore \bunsuu52x=10$$
$$\therefore x=4,\,y=3$$
$$\therefore \bm{P(4,\,3)}$$
\end{enumerate}

{\bf 【コメント】}

どのような解き方でもかまいませんが,解けるように。

(2)は反射です。代数的にやろうとすると,文系は難しいと思う。(理系でも大変です)
\end{multicols*}


\end{caprm}%8
\newpage

%% 57 %%
\Name{数学IAIIB応用}
\hakosyokika
\begin{caprm}
座標平面上の2点Q$(1,\,1)$,R$\left(2,\,\bunsuu12\right)$に対して,点Pが円$x^{2}+y^{2}=1$の周上を動くとき,
\begin{enumerate}
\item \sankaku{PQR}の重心Gの軌跡を求めよ。
\item 点Pから\sankaku{PQR}の重心Gまでの距離が最小となるとき,点Pの座標を求めよ。

\item \sankaku{PQR}の面積の最小値を求めよ。

\end{enumerate}



\begin{multicols*}{2}
{\bf 【解答】}

\begin{enumerate}
\item $P(s,\,t)$,$G(x,\,y)$とおくと,
$$x=\bunsuu{s+1+2}3,\,y=\bunsuu{t+1+1/2}3$$
$$\therefore s=3x-3,\,t=3y-\bunsuu32$$
$s^2+t^2=1$に代入して,
$$(3x-3)^2+\left(3y-\bunsuu32\right)^2=1$$
$$(x-1)^2+\left(y-\bunsuu12\right)^2=\bunsuu19$$
よって点Gの軌跡は,
$$\bm{\text{\bf 中心}\left(1,\,\bunsuu12\right)\text{\bf ,半径}\bunsuu13\text{\bf の円}}$$

\item $s=\cos\theta$,$t=\sin\theta$とおくと,
\begin{align*}
PG^2&=(x-s)^2+(y-t)^2\\
&=\left(1-\bunsuu23s\right)^2+\left(\bunsuu12-\bunsuu23t\right)^2\\
&=1-\bunsuu43s+\bunsuu49s^2+\bunsuu14-\bunsuu23t+\bunsuu49t^2\\
&=\bunsuu54+\bunsuu49-\bunsuu23(2s+t)\quad (\because s^2+t^2=1)\\
&=\bunsuu{61}{36}-\bunsuu{2\sqrt5}3\left(\bunsuu2{\sqrt 5}\cos\theta+\bunsuu1{\sqrt 5}\sin\theta\right)\\
&=\bunsuu{61}{36}-\bunsuu{2\sqrt5}3\sin(\theta+\alpha)\\
&\quad \quad \left(\cos\alpha=\bunsuu1{\sqrt 5},\,\sin\alpha=\bunsuu2{\sqrt 5}とする\right)
\end{align*}
よって,$\sin(\theta+\alpha)=1$となるとき,$PG^2$,すなわちPGは最小となる。
$$\min\text{PG}=\sqrt{\bunsuu{61}{36}-\bunsuu{2\sqrt5}3}=\bunsuu{3\sqrt5-4}6$$
これは
$\cos\theta=\bunsuu2{\sqrt 5},\,\sin\theta=\bunsuu1{\sqrt 5}$のときで,
$$\therefore \bm{P\left(\bunsuu2{\sqrt 5},\,\bunsuu1{\sqrt 5}\right)}$$

\item $\bekutoru{PQ}=\left(1-\cos\theta,\,1-\sin\theta\right)$,

$\bekutoru{PR}=\left(2-\cos\theta,\,\bunsuu12-\sin\theta\right)$とおくと,
\begin{align*}
\sankaku{PQR}
&=\bunsuu12\Big|\left(1-\cos\theta\right)\left(\bunsuu12-\sin\theta\right)\\
&\qquad\quad -\left(1-\sin\theta\right)\left(2-\cos\theta\right)\Big|\\
&=\bunsuu12\Big|\bunsuu12-\sin\theta-\bunsuu12\cos\theta+\sin\theta\cos\theta\\
&\qquad\quad -2+2\sin\theta+\cos\theta-\sin\theta\cos\theta\Big|\\
&=\bunsuu12\Big|-\bunsuu32+\sin\theta+\bunsuu12\cos\theta\Big|\\
&=\bunsuu{\zettaiti{-3+\cos\theta+2\sin\theta}}4\\
&=\bunsuu{\zettaiti{-3+\sqrt{5}\cos(\theta-\alpha)}}4
\geqq \bunsuu{3-\sqrt 5}4
\end{align*}
よって,\sankaku{PQR}の面積は$P\left(\bunsuu1{\sqrt 5},\,\bunsuu2{\sqrt 5}\right)$で
$\bm{\text{\bf 最小値 } \bunsuu{3-\sqrt 5}4}$
をとる。
\end{enumerate}

{\bf 【コメント】}

(2)以降は重ためです。計算方法はいろいろありますが今回は三角関数でやりました。
(2)(3)を図形的に考えるには下の図を参考に。
まず,(1)で求めた円と単位円は相似で,その相似の中心はQRの中点Mです。(2)のPQが最小となるときのPとGの位置は$P_2$,$G_2$となります。(3)は直線QRと平行な単位円の接線のうち,近い方が$P=P_3$となります。

\begin{center}
\begin{zahyou}[haiti=t,ul=12mm,yokozikukigou={$x$},tatezikukigou={$y$},gentenhaiti={[sw]},yokozikuhaiti={[s]},tatezikuhaiti={[w]},yscale=1](-1.5,3)(-1.5,2)

\tenretu{Q(1,1)n;R(2,.5)n}
\tenretu*{A(1,.5)}

\En\O{1}

\calcval{1/3}\ra

\calcval{-2*(sqrt(3))}\xb
\En\A\ra

\Bunten\Q\R11\M

\Tyokusen\Q\R{}{}
\Put\M[ne]{M}

\Drawline{\M\O}


\CandL\O{1}\O\M\PP\P

\CandL\O{1}\O{(1,2)}\PPP\PPPP

\Put\P[w]{$P_2$}

\Put\PPPP[n]{$P_3$}

\CandL\A\ra\O\M\GG\G

\mTyokusen\PPPP{(-2,1)}{}{}

\Put\G[se]{$G_2$}

\kuromaru{\Q;\R;\M;\P;\G;\PPPP}

%\thicklines
%\YGurafu\Gx{0}\xmax

%\YGurafu\Hx{0}\xmax



%\thinlines
%\YTen\Fx{1}\A
%\YTen\Fx{-1}\B

%\Put\A[syaei=xy,xlabel=1,ylabel=\frac12]{}
%\Put\B[syaei=xy,xlabel=-1,ylabel=-\frac12]{}

%\Put{(2,0)}[se]{$2$}
%\Put{(3,0)}[se]{$3$}
%\Put{(4,0)}[se]{$4$}

%\Drawline{(3,\ymax)(3,\ymin)}
%\Drawline{(4,\ymax)(4,\ymin)}
%\Drawline{(\Pii,\ymax)(\Pii,\ymin)}

\end{zahyou}
\end{center}


\end{multicols*}



\end{caprm}
%14
\newpage

%% 58 %%
\Name{数学IAIIB応用}
\hakosyokika
正の数$a$,$b$,$x$,$y$を考える。$a+b=1$ならば,すべての自然数$n$について不等式$(ax+by)^{n}\leqq ax^{n}+by^{n}$であることを証明せよ。
%37



\begin{multicols*}{2}
{\bf 【解答】}

\begin{enumerate}[I)]

\item $n=1$のとき,両辺ともに,$ax+by$となり,等号が成り立つ。
\item $n=k$のとき,
$$(ax+by)^{k}\leqq ax^{k}+by^{k}\cdots☆$$
の成立を仮定する。

$n=k+1$のとき,

\begin{align*}
&(ax+by)^{k+1}=(ax+by)(ax+by)^{k}\\
&\leqq (ax+by)(ax^k+by^k)\quad (\because ax+by>0,\,☆)\\
&=a^2x^{k+1}+abxy^k+abx^ky+b^2y^{k+1}=\maru1
\end{align*}
\begin{align*}
 & ax^{k+1}+by^{k+1}-(ax+by)^{k+1}\\
&\geqq ax^{k+1}+by^{k+1}-\maru1\\
&=a(1-a)x^{k+1}-abxy^k-abx^ky+b(1-b)y^{k+1}\\
&=ab(x^{k+1}-xy^k-x^ky+y^{k+1})\quad (\because a+b=1)\\
&=ab(x-y)(x^k-y^k)\\
&\geqq 0\quad (\because x-yとx^k-y^kは0または同符号)
\end{align*}
$$\therefore (ax+by)^{k+1}\leqq ax^{k+1}+by^{k+1}$$
よって,$n=k+1$のとき不等式は成立する。
\end{enumerate}

I),II),帰納的に,不等式$(ax+by)^{n}\leqq ax^{n}+by^{n}$は自然数$n$について成り立つ。

{\bf 【コメント】}

後半は次のように書いてもよいでしょう。
\begin{align*}
&ab(x-y)(x^k-y^k)\\
&=ab(x-y)^2(x^{k-1}+x^{k-2}y+\cdots+xy^{y-2}+y^{k-1})\\
&\geqq 0
\end{align*}

\columnbreak
この不等式は『グラフの凸性』より,示すことができます。

$x\geqq 0$で,$y=f(x)=x^n$のグラフが下に凸であるので,グラフ上の2点$(x,\,f(x))$,$(y,\,f(y))$を$b:a$に内分する点Pはグラフの上側にある。


\begin{center}
\begin{zahyou}[ul=20mm,gentenhaiti={[se]},tatezikukigou={},yokozikukigou={},](-.5,3.1)(-.5,2.5)
 \def\Fx{(X**2)/4}

   \YGurafu\Fx{0}\xmax
   \YTen\Fx{1}\A
   \YTen\Fx{3}\B
   \YTen\Fx{2}\Q
   \Bunten\A\B11\P

 \Put\A[syaei=xy,xlabel=x,ylabel=f(x)]{}
 \Put\B[syaei=xy,xlabel=y,ylabel=f(y)]{}
 \Put\P[syaei=xy,xlabel=\frac{ax+by}{a+b},ylabel=\frac{af(x)+bf(y)}{a+b}]{}
 \Put\P[nw]{P}

\Put\Q[syaei=y,ylabel=f\left(\frac{ax+by}{a+b}\right)]{}
\Hen_ko\B\P{$a$}
\Hen_ko\P\A{$b$}

 \Drawline{\A\B}


\end{zahyou}
 \end{center}


よって,
$$f\left(\bunsuu{ax+by}{a+b}\right)\leqq \bunsuu{af(x)+bf(y)}{a+b}$$

$a+b=1$,$f(x)=x^n$より,
$$\therefore (ax+by)^{n}\leqq ax^{n}+by^{n}$$

が成り立つ。


\end{multicols*}
\newpage

%% 59 %%
\Name{数学IAIIB応用}
\hakosyokika
\begin{caprm}
放物線${\it C}:y=\bunsuu14x^{2}$と,点B$(0,\,b)$を中心とする半径$r$の円(ただし,$r<b$)は異なる2点$A_{1}$,$A_{2}$を共有し,それ以外には共有点をもたないとする。ここで,点$A_{1}$,$A_{2}$はそれぞれ,第1象限,第2象限にあるとする。

\begin{enumerate}
\item 円の中心Bおよび2点$A_{1}$,$A_{2}$の座標を
$r$を用いて表せ。さらに,$r$の値の範囲を求めよ。

\item $r=4$のとき,$\kaku{}A_{1}BA_{2}$の大きさを求めよ。ただし,$0\Deg<\kaku{}A_{1}BA_{2}<180\Deg$とする。



\item (2)で求めた$\kaku{}A_{1}BA_{2}$に対する弧$\ko*{A_{1}A_{2}}$と放物線${\it C}$で囲まれた図形の面積を求めよ。

\end{enumerate}



\begin{multicols*}{2}
{\bf 【解答】}

\begin{enumerate}
\item 放物線${\it C}$と円$x^2+(y-b)^2=r^2\cdots\maru1$の2式より,$x$を消去した$y$の2次方程式が重解を持てばよい。

\begin{center}
\begin{zahyou}[haiti=t,ul=5mm,yokozikukigou={$x$},tatezikukigou={$y$},gentenhaiti={[sw]},yokozikuhaiti={[s]},tatezikuhaiti={[w]},yscale=1](-5,5)(-.5,10)
\def\Fx{1/4*X**2}

\YGurafu*\Fx
\calcval{2*(sqrt(3))}\xa
\calcval{-2*(sqrt(3))}\xb

\tenretu{B(0,5)}
\tenretu*{A(\xa,3);C(\xb,3)}

\Put\A[e]{$A_1$}
\Put\C[w]{$A_2$}

\En\B{4}
\Hasen{\A\C}
\kuromaru{\B}
%\thicklines
%\YGurafu\Gx{0}\xmax

%\YGurafu\Hx{0}\xmax



%\thinlines
%\YTen\Fx{1}\A
%\YTen\Fx{-1}\B

%\Put\A[syaei=xy,xlabel=1,ylabel=\frac12]{}
%\Put\B[syaei=xy,xlabel=-1,ylabel=-\frac12]{}

%\Put{(2,0)}[se]{$2$}
%\Put{(3,0)}[se]{$3$}
%\Put{(4,0)}[se]{$4$}

%\Drawline{(3,\ymax)(3,\ymin)}
%\Drawline{(4,\ymax)(4,\ymin)}
%\Drawline{(\Pii,\ymax)(\Pii,\ymin)}

\end{zahyou}
\end{center}


放物線${\it C}$の式より$x^2=4y$として,\maru1に代入し,
$$4y+y^2-2by+b^2-r^2=0$$
$$\therefore y^2-2(b-2)y+b^2-r^2=0$$
$$D/4=(b-2)^2-(b^2-r^2)=-4b+4+r^2=0$$
$$\therefore b=1+\bunsuu{r^2}4\quad \therefore \bm{B\left(0,\,1+\bunsuu{r^2}4\right)}$$
このとき,$A_1$,$A_2$の$y$座標は等しく,
$$y=b-2=\bunsuu{r^2}4-1$$
放物線${\it C}$の式より,
$$x^2=r^2-4\quad \therefore x=\pm\sqrt{r^2-4}$$
\begin{align*}
\therefore \quad&\bm{A_1\left(\sqrt{r^2-4},\,\bunsuu{r^2}4-1\right)}\\
&\bm{A_2\left(-\sqrt{r^2-4},\,\bunsuu{r^2}4-1\right),\quad r>2}
\end{align*}

\item $r=4$のとき,$A_1(2\sqrt 3,\,3),\,A_2(-2\sqrt 3,\,3),\,B(0,\,5)$である。

\begin{center}
\begin{zahyou}[haiti=t,ul=5mm,yokozikukigou={$x$},tatezikukigou={$y$},gentenhaiti={[sw]},yokozikuhaiti={[s]},tatezikuhaiti={[w]},yscale=1](-5,5)(-.5,10)
\def\Fx{1/4*X**2}
\def\Gx{5-(sqrt(16-X**2))}

\YGurafu*\Fx
\calcval{2*(sqrt(3))}\xa
\calcval{-2*(sqrt(3))}\xb

\tenretu{B(0,5)nw}
\tenretu*{A(\xa,3);C(\xb,3);D(0,3)}

\Put\A[e]{$A_1$}
\Put\C[w]{$A_2$}
\Put\B[ne]{$5$}
\Put\D[se]{$3$}

\En\B{4}
\Drawline{\A\C\B\A}
\Hen_ko\B\C{$4$}
\Hen_ko\C\D{$2\sqrt{3}$}
\Hen_ko\D\B{$2$}


\Put\A[syaei=x,xlabel=2\sqrt 3]{}
\Put\C[syaei=x,xlabel=-2\sqrt 3]{}

\kuromaru{\B}
%\thicklines
%\YGurafu\Gx{0}\xmax

%\YGurafu\Hx{0}\xmax


\YNurii*\Gx\Fx\xb\xa
%\thinlines
%\YTen\Fx{1}\A
%\YTen\Fx{-1}\B

%\Put\A[syaei=xy,xlabel=1,ylabel=\frac12]{}
%\Put\B[syaei=xy,xlabel=-1,ylabel=-\frac12]{}

%\Put{(2,0)}[se]{$2$}
%\Put{(3,0)}[se]{$3$}
%\Put{(4,0)}[se]{$4$}

%\Drawline{(3,\ymax)(3,\ymin)}
%\Drawline{(4,\ymax)(4,\ymin)}
%\Drawline{(\Pii,\ymax)(\Pii,\ymin)}

\end{zahyou}
\end{center}


図より,$\bm{\kaku{}A_1BA_2=120\Deg}$

\item 放物線${\it C}$と直線$y=3$で囲まれた部分$(A)$から,扇形$A_1BA_2$を除き,$\sankaku{}A_1BA_2$を加えればよい。(上図の斜線部分)

\begin{align*}
&(A)-扇形A_1BA_2+\sankaku{}A_1BA_2\\
&=4\sqrt3\cdot 3\cdot \bunsuu23-4^2\pi\cdot\bunsuu13+4^2\sin120\Deg\cdot\bunsuu12\\
&=8\sqrt 3-\bunsuu{16}3\pi+4\sqrt 3\\
&=\bm{12\sqrt 3-\bunsuu{16}{3}\pi}
\end{align*}

\end{enumerate}

{\bf 【コメント】}

(3)の部分(A)の面積は$\dint{-2\sqrt 3}{2\sqrt 3}\left(3-\bunsuu{x^2}4\right)dx$ですが,横が$4\sqrt 3$,縦が$3$の長方形の$\bunsuu23$倍として計算できます。
\end{multicols*}



\end{caprm}%15

\newpage

%% 60 %%
\Name{数学IAIIB応用}
\hakosyokika
任意の実数$x$に対して常に$ax^{2}+2bx+c>px^{2}+2qx+r>0\cdots☆$
が成立することき,$ac-b^{2}>pr-q^{2}$が成立することを示せ。ただし,$a\neqq p$,$ap\neqq 0$とする。
%59


\begin{multicols*}{2}
{\bf 【解答】}

$f(x)=ax^{2}+2bx+c$,$g(x)=px^{2}+2qx+r$とおく。

つねに,$f(x)-g(x)>0\wedge g(x)>0$が成り立つので,
$$a-p>0\wedge p>0\quad \therefore a>p>0\cdots\maru1$$

また,$f(x)=a\left(x+\bunsuu ba\right)^2+\bunsuu{ac-b^2}{a}$,$a>0$であるから,
$$f(x)の最小値は\bunsuu{ac-b^2}{a}$$

同様に,
$$g(x)の最小値は\bunsuu{pr-q^2}{p}$$
$f(x)の最小値>g(x)の最小値>0$より,
$$\bunsuu{ac-b^2}{a}>\bunsuu{pr-q^2}{p}>0\cdots\maru2$$

\maru1,\maru2より,
$$\bunsuu{ac-b^2}{a}\cdot a>\bunsuu{pr-q^2}{p}\cdot p$$
$$\therefore ac-b^{2}>pr-q^{2}$$

{\bf 【コメント】}

$y=f(x)$のグラフと$y=g(x)$のグラフの位置関係より,2つの不等式を作れば示せます。\maru1は放物線の開き具合を不等式で表しています。ここは図は明らかですが,説明となると難しかったのではないでしょうか。上の解答は明快です。参考に。
\end{multicols*}
\newpage




%%61-65 2008/8%%
%% 61 %%
\Name{数学IAIIB応用}
\hakosyokika
数列$\{a_{n}\}$が$a_{1}=2$,$(n+2)a_{n+1}-na_{n}=3n+4\quad (n\geqq 1)$で定義されるとき,
\begin{enumerate}
\item $a_{2}$,$a_{3}$,$a_{4}$を求めよ。

\item $a_{n}$を$n$の式で表せ。
\end{enumerate}%4

\begin{multicols*}{2}
{\bf 【解答】}

\begin{enumerate}
\item 漸化式で$n=1$として,
$$3a_2-a_1=7$$
$$\therefore \bm{a_2}=\bunsuu{7+a_1}{3}=\bunsuu93=\bm{3}$$

漸化式で$n=2$として,
$$4a_3-2a_2=10$$
$$\therefore \bm{a_3}=\bunsuu{10+2a_2}{4}=\bunsuu{16}4=\bm{4}$$

漸化式で$n=3$として,
$$5a_4-3a_3=13$$
$$\therefore \bm{a_4}=\bunsuu{13+4a_3}{5}=\bunsuu{25}5=\bm{5}$$


\item $a_{n}=n+1\cdots☆$と推測される。

\item[(I)] $n=1$のとき,$a_1=2$より,☆は正しい。
\item[(II)] $n=k$のとき,☆の成立を仮定する。

漸化式で$n=k$として,
$$(k+2)a_{k+1}-ka_{k}=3k+4$$
仮定より,
$$(k+2)a_{k+1}-k(k+1)=3k+4$$
$$\therefore (k+2)a_{k+1}=k^2+4k+4$$
$$\therefore (k+2)a_{k+1}=(k+2)^2$$
$k+2\neqq 0$より,$a_{k+1}=k+2$となり,$n=k+1$のとき☆は成り立つ。

(I)(II),帰納的に自然数$n$について☆は成り立つ。
$$\bm{a_n=n+1}$$
\end{enumerate}%4
\columnbreak



{\bf 【コメント】}
漸化式を解いてみましょう。

漸化式の両辺を$(n+1)$倍して,
$$(n+2)(n+1)a_{n+1}-(n+1)na_{n}=(3n+4)(n+1)$$
$n\geqq 2$として,
\begin{align*}
(n+1)na_{n}&=2\cdot1\cdot2+\tretuwa{k=1}{n-1}(3k+4)(k+1)\\
&=4+\tretuwa{k=2}{n}(3k+1)k\\
&=\tretuwa{k=1}{n}(3k^2+k)\\
&=\bunsuu12n(n+1)(2n+1)+\bunsuu12n(n+1)\\
&=n(n+1)^2\quad (n=1も成立)
\end{align*}
$$\therefore (n+1)na_{n}=n(n+1)^2$$
両辺を$n(n+1)\,(\neqq0)$で割って,
$$\therefore \bm{a_n=n+1}$$
\end{multicols*}
\newpage


%% 62 %%
\Name{数学IAIIB応用}
\hakosyokika
3の倍数でない自然数を小さいものから順に並べたとき,$n$番目の数を$a_{n}$とする。例えば,$a_{1}=1$,$a_{2}=2$,$a_{3}=4$,$a_{4}=5$,$a_{5}=7$,$a_{6}=8$,$a_{7}=10$,$a_{8}=11$,$\cdots$である。
\begin{enumerate}
\item $a_{20}$を求めよ。

\item $\tretuwa{k=1}{20}a_{k}$の値を求めよ。

\item $a_{n}=2008$となる$n$の値を求めよ。
\end{enumerate}
%8

\begin{multicols*}{2}
{\bf 【解答】}

\begin{enumerate}
\item $\bm{a_{20}}=3\times \bunsuu{20}{2}-1=\bm{29}$


\item $\begin{aligned}[t]
\tretuwa{k=1}{20}a_{k}&=\tretuwa{k=1}{30}k-\tretuwa{k=1}{10}3k\\
&=\bunsuu{30\times31}{2}-\bunsuu{3\times10\times11}{2}\\
&=\bunsuu{30(31-11)}{2}\\
&=\bm{300}
\end{aligned}$

\item $2007=3\cdot669$より,
$$\bm{n}=669\times 2+1=\bm{1339}$$
\end{enumerate}%4



{\bf 【コメント】}

$a_n$の一般式にこだわらなくてもよいと思いますが,一応,書いておきます。

$a_n$の階差数列$\{b_n\}$は$1,2,1,2,\cdots$となるので,
$$b_n=\bunsuu{3+(-1)^n}2$$
$n\geqq 2$として,
\begin{align*}
a_n&=1+\tretuwa{k=1}{n-1}b_k\\
&=1+\tretuwa{k=1}{n-1}\left(\bunsuu32-\bunsuu12(-1)^{k-1}\right)\\
&=1+\bunsuu32(n-1)-\bunsuu12\cdot\bunsuu{1-(-1)^{n-1}}{1-(-1)}\\
&=\bunsuu32n-\bunsuu34+\bunsuu{(-1)^{n-1}}4\quad(n=1も成立)
\end{align*}
これを使うと,(2)(3)がすぐ解けるという訳ではありません。(かえって面倒です)


\end{multicols*}
\newpage

%% 63 %%
\Name{数学IAIIB応用}
\hakosyokika

$0\leqq x<\pi$のとき,関数$f(x)=a\cos^{2}x+b\sin^{2}x+\cos x\sin x$について,次の問いに答えなさい。ただし,$a$,$b$は実数である。
\begin{enumerate}
\item $f(x)$を$\cos 2x$と$\sin 2x$を用いて表せ。

\item $f(x)$の最大値と最小値をそれぞれ$a$,$b$を用いて表せ。

\item $f(x)$の最大値と最小値がをそれぞれ$2$,$-1$であるとき,$a$,$b$の値を求めよ。
\end{enumerate}
%36



\begin{multicols*}{2}
{\bf 【解答】}

\begin{enumerate}
\item 2倍角の公式より,
\begin{align*}
\bm{f(x)}&=a\cdot\bunsuu{1+\cos2x}2+b\cdot\bunsuu{1-\cos2x}2+\bunsuu12\sin2x\\
&=\bm{\bunsuu12\{(a-b)\cos2x+\sin2x+a+b\}}\end{align*}


\item 合成して,
\begin{align*}
f(x)&=\bunsuu12\{\sqrt{(a-b)^2+1}\sin(2x+\alpha)+a+b\}\end{align*}
$\alpha\leqq 2x+\alpha<2\pi+\alpha$より,1周期分あるので,
$$\begin{emcases}
&\bm{最大値 \bunsuu12\{\sqrt{(a-b)^2+1}+a+b\}}\\ \\
&\bm{最小値 \bunsuu12\{-\sqrt{(a-b)^2+1}+a+b\}}\end{emcases}$$

\item $最大値=2\cdots\maru1,\,最大値=-1\cdots\maru2$

$\maru1+\maru2$より,
$$a+b=1\cdots\maru3$$

\maru1より,
$$\sqrt{(a-b)^2+1}+1=4\quad \therefore (a-b)^2=8$$
$$\therefore a-b=\pm2\sqrt 2\cdots\maru4$$

\maru3,\maru4より,
$$\bm{(a,\,b)=\left(\bunsuu{1\pm2\sqrt 2}2,\,\bunsuu{1\mp2\sqrt 2}2\right) (複合同順)}$$
\end{enumerate}%4



{\bf 【コメント】}

三角関数の2倍角の公式と合成です。合成は$\sin$,$\cos$どちらでも自由自在にできるように。また,(3)で\maru1,\maru2より,$a$,$b$を求めるときにはちょっとかわった連立方程式を解きますが,上記の解法を参考にしてください。


\end{multicols*}

\newpage



%% 64 %%
\Name{数学IAIIB応用}
\hakosyokika
\begin{caprm}
平面上に\sankaku{ABC}がある。この平面上の点Pに対してAPの中点をQ,BQの中点をR,CRの中点をSとする。2点P,Sが一致しているとき,
\begin{enumerate}
\item $\bekutoru{AB}=\Beku b$,$\bekutoru{AC}=\Beku c$とするとき,$\bekutoru{AP}$を$\Beku b$,$\Beku c$を用いて表せ。

\item \sankaku{PQR}と\sankaku{ABC}の面積比を求めよ。
\end{enumerate}



\begin{multicols*}{2}
{\bf 【解答】}

\begin{enumerate}
\item $\bekutoru{AP}=x\Beku b+y\Beku c\cdots\maru1$とおくと,
\begin{align*}
\bekutoru{AQ}&=\bunsuu12\bekutoru{AP}=\bunsuu x2\Beku b+\bunsuu y2\Beku c
\end{align*}

\begin{align*}
\therefore \bekutoru{AR}&=\bunsuu{\bekutoru{AB}+\bekutoru{AQ}}2
=\bunsuu{\Beku b+\frac x2\Beku b+\frac y2\Beku c}2\\
&=\bunsuu{x+2}4\Beku b+\bunsuu y4\Beku c
\end{align*}

\begin{align*}
\therefore \bekutoru{AS}&=\bunsuu{\bekutoru{AR}+\bekutoru{AC}}2
=\bunsuu{\frac {x+2}4\Beku b+\frac y4\Beku c+\Beku c}2\\
&=\bunsuu{x+2}8\Beku b+\bunsuu {y+4}8\Beku c\cdots\maru2
\end{align*}

$P=S$より,$\bekutoru{AP}=\bekutoru{AS}$。

\maru1,\maru2,$\Beku b$と$\Beku c$は一次独立より,
$$x=\bunsuu{x+2}8,\,y=\bunsuu{y+4}8$$
$$\therefore x=\bunsuu27,\,y=\bunsuu47$$
$$\therefore \bm{\bekutoru{\bf AP}=\bunsuu27\Beku b+\bunsuu47\Beku c}$$


\item APの延長とBCが交わる点をTとする。

\begin{center}
\begin{zahyou*}[haiti=t,ul=12mm,yokozikukigou={$x$},tatezikukigou={$y$},gentenhaiti={[sw]},yokozikuhaiti={[s]},tatezikuhaiti={[e]}](-.5,4.5)(-.5,3.5)
\tenretu{A(1,3)n;B(0,0)w;C(4,0)e}

\Bunten\B\C21\T
\Bunten\A\T61\P
\Bunten\A\T34\Q
\Bunten\C\P2{-1}\R


\Put\T[s]{T}
\Put\P[ne]{P}
\Put\Q[e]{Q}
\Put\R[nw]{R}
\Takakkei{\A\B\C}
\Drawlines{\A\T;\B\Q;\C\R}
\end{zahyou*}
\end{center}



$\bekutoru{AT}=k\bekutoru{AP}$とおくと,
$$\bekutoru{AT}=\bunsuu{2k}7\Beku b+\bunsuu{4k}7\Beku c$$
TはBC上より,
$$\bunsuu{2k}7+\bunsuu{4k}7=1\quad \therefore k=\bunsuu76$$

よって,$\bekutoru{AT}=\bunsuu76\bekutoru{AP}$となり,
$$AQ:QP:PT=3:3:1,\,BT:TC=2:1$$

$\sankaku{ABC}=1$とおくと,
$$\sankaku{APC}=\bunsuu26\cdot\bunsuu67=\bunsuu27$$
他も同様なので,
$$\sankaku{APC}=\sankaku{BQA}=\sankaku{CRB}=\bunsuu27$$
$$\therefore \sankaku{PQR}=1-\bunsuu72\times 3=\bunsuu17$$
$$\therefore \bm{\sankaku{\bf PQR}:\sankaku{\bf ABC}=1:7}$$
\end{enumerate}



{\bf 【コメント】}

下の図の7つの三角形の面積は等しくなります。
\begin{center}
\begin{zahyou*}[haiti=t,ul=10mm,yokozikukigou={$x$},tatezikukigou={$y$},gentenhaiti={[sw]},yokozikuhaiti={[s]},tatezikuhaiti={[e]}](-.5,4.5)(-.5,3.5)
\tenretu{A(1,3)n;B(0,0)w;C(4,0)e}

\Bunten\B\C21\T
\Bunten\A\T61\P
\Bunten\A\T34\Q
\Bunten\C\P2{-1}\R


%\Put\T[s]{T}
\Put\P[ne]{P}
\Put\Q[e]{Q}
\Put\R[nw]{R}
\Takakkei{\A\B\C}
\Drawlines{\A\P;\B\Q;\C\R;\A\R;\B\P;\C\Q}
\end{zahyou*}
\end{center}
$$\therefore \bm{\sankaku{\bf PQR}:\sankaku{\bf ABC}=1:7}$$

$$\therefore \sankaku{PBC}:\sankaku{PCA}:\sankaku{PAB}=1:2:4$$

$$\therefore \Beku p=\bunsuu{\Beku a+2\Beku b+4\Beku c}{1+2+4}$$
Aが基点なので,$ \bm{\bekutoru{\bf AP}=\bunsuu27\Beku b+\bunsuu47\Beku c}$

\end{multicols*}

\end{caprm}
%63

\newpage



%% 65 %%
\Name{数学IAIIB応用}
\hakosyokika
自然数$n$に対して,$a_{n}=(\cos2^{n})(\cos2^{n-1})\cdots(\cos2)(\cos1)$とおく。ただし,角の大きさは弧度法を用いる。
\begin{enumerate}
\item $a_{1}=\bunsuu{\sin4}{4\sin1}$を示せ。
\item $a_{n}=\bunsuu{\sin2^{n+1}}{2^{n+1}\sin1}$を示せ。
\item $a_{n}<\bunsuu{\sqrt 2}{2^{n+1}}$を示せ。
\end{enumerate}


%37


\begin{multicols*}{2}
{\bf 【解答】}

\begin{enumerate}
\item 2倍角の公式繰り返し用いて,
\begin{align*}
\sin4&=2\sin2\cos2\\
&=2(2\sin1\cos1)\cos2\\
\therefore a_1&=\cos2\cos1=\bunsuu{\sin4}{4\sin1}
\end{align*}


\item $a_{n}=\bunsuu{\sin2^{n+1}}{2^{n+1}\sin1}\cdots☆$を帰納法で示す。


\item[(i)] $n=1$のときは(1)より,☆は正しい。
\item[(ii)] $n=k$のとき,☆が成り立つと仮定する。

$n=k+1$のとき,
\begin{align*}
a_{k+1}&=\cos2^{k+1}a_k\\
&=\bunsuu{\cos2^{k+1}\sin2^{k+1}}{2^{k+1}\sin1}\quad(\because 仮定)\\
&=\bunsuu{\frac12\sin(2\cdot2^{k+1})}{2^{k+1}\sin1}\quad(\because 2倍角の公式)\\
&=\bunsuu{\sin2^{k+2}}{2^{k+2}\sin1}
\end{align*}
よって,☆は成り立つ。

(i),(ii),帰納的に自然数$n$について☆は成り立つ。
$$a_{n}=\bunsuu{\sin2^{n+1}}{2^{n+1}\sin1}$$


\item $\bunsuu{\sin2^{n+1}}{\sin1}<\sqrt2$であればよい。

$\bunsuu{\pi}4<1<\bunsuu{\pi}{3}$より,
$$\sin \bunsuu{\pi}4<\sin1<\bunsuu{\pi}{3}$$
$$\therefore \bunsuu1{\sqrt 2}<\sin1<\bunsuu{\sqrt 3}2$$
$$\therefore \bunsuu2{\sqrt 3}<\bunsuu{1}{\sin1}<\sqrt 2$$
$\sin2^{n+1}\leqq 1$より,
$$\therefore \bunsuu{\sin2^{n+1}}{\sin1}<\sqrt 2$$
$$\therefore a_n=\bunsuu{\sin2^{n+1}}{2^{n+1}\sin1}<\bunsuu{\sqrt 2}{2^{n+1}}$$

与えられた不等式は示された。


\end{enumerate}%4


{\bf 【コメント】}


$\bunsuu{\pi}4<1<\bunsuu{\pi}{3}$でもわかるように,弧度法で1は結構大きな角です。約$57.29578\Deg$で$\bunsuu{\pi}3$に近いこともわかります。


\end{multicols*}


\newpage





%%66-70 2008/9%%
%% 66 %%
\Name{数学IAIIB応用}
\hakosyokika
6人の学生を3組に分ける。まず,3人,2人,1人の3組に分ける方法は\Hako 通りある。次に2人ずつの3組に分ける方法は\Hako 通りある。6人の学生を3組に分ける方法は全部で\Hako 通りある。
%8


\begin{multicols*}{2}
{\bf 【解答】}

\begin{enumerate}[(ア)]
\item $\kumiawase63\cdot\kumiawase32\cdot\kumiawase11=\bm{60(通り)}$
\item $\bunsuu{\kumiawase62\cdot\kumiawase42\cdot\kumiawase22}{3\kaizyou}=\bm{15(通り)}$
\item 人数の分け方は次の3パターン$$(1,\,1,\,4)(1,\,2,\,3)(2,\,2,\,2)$$


\begin{enumerate}[(i)]
\item $(1,\,1,\,4)$のとき,
$$\bunsuu{\kumiawase61\cdot\kumiawase51\cdot\kumiawase44}{2\kaizyou}=15(通り)$$
\item $(1,\,2,\,3)$のとき,(ア)より60通り。
\item $(2,\,2,\,2)$のとき,(イ)より15通り。

\end{enumerate}
以上,合計して,
$$15+60+15=\bm{90(通り)}$$


\end{enumerate}%4



{\bf 【コメント】}

『組み分け問題』です。組の区別はしっかりマスターしましたか?

(3)の別解

6人がA,B,Cの組を選択($3\kaizyou$通り)。ただし,1組に集中するケース($3$通り)と2組に集中するケース($\kumiawase32(2^6-2)$通り)を除く。最後に,A,B,Cの組の別をなくす($3\kaizyou$で割る)。

\begin{align*}
\bunsuu{\{3^6-\kumiawase32(2^6-2)-3\}}{3\kaizyou}
&=\bunsuu{729-3\cdot62-3}6\\
&=\bunsuu{540}6\\
&=\bm{90(通り)}
\end{align*}
\end{multicols*}



\newpage


%% 67 %%
\Name{数学IAIIB応用}
\hakosyokika
1,2,3,4を重複を許して並べてできる数について,
\begin{enumerate}

\item 各桁の数の和が6となる5桁の数の個数を求めよ。

\item 各桁の数の和が7となる5桁の数の個数を求めよ。

\item 各桁の数の和が$k+3$となる$k$桁の数の個数を求めよ。



\end{enumerate}

%8


\begin{multicols*}{2}
{\bf 【解答】}

\begin{enumerate}
\item $1,\,1,\,1,\,1,\,2$を並べて,
$$\kumiawase51=\bm{5(個)}$$

\item
\begin{enumerate}[(i)]
\item $1,\,1,\,1,\,1,\,3$を並べて,
$$\kumiawase51=5(個)$$
\item $1,\,1,\,1,\,2,\,2$を並べて,
$$\kumiawase52=10(個)$$
\end{enumerate}
合計して,
$$5+10=\bm{15(個)}$$

\item \begin{enumerate}[(i)]
\item $\underbrace{1,\,\cdots\cdots\cdots,\,1}_{k-1個},\,4$を並べて,
$$\kumiawase k1=k(個)$$

\item $\underbrace{1,\,\cdots\cdots,\,1}_{k-2個},\,2,\,3$を並べて,
$$\kumiawase k1\cdot\kumiawase{n-1}1=k(k-1)(個)$$

\item $\underbrace{1,\,\cdots,\,1}_{k-3個},\,2,\,2,\,2$を並べて,
$$\bunsuu{k\kaizyou}{3\kaizyou(k-3)\kaizyou}=\bunsuu{k(k-1)(k-2)}{6}(個)$$


\end{enumerate}
合計して,
\begin{align*}
{\rm (i)+(ii)+(iii)}&=k+k(k-1)+\bunsuu{k(k-1)(k-2)}{6}\\
&=\bunsuu16k(6+6k-6+k^2-3k+2)\\
&=\bunsuu16k(k^2+3k+2)\\
&=\bm{\bunsuu16k(k+1)(k+2)(個)}
\end{align*}

\end{enumerate}



{\bf 【コメント】}

(3)でいきなり文字$k$が入ってきます。ちゃんと数えられましたか?

答えは$\kumiawase{k+2}3$なので,別の意味付けも可能です。

例えば,$i$桁目の数字を$a_i$とし,$b_j=\tretuwa{i=1}{j}a_i$とすると,
$$1\leqq b_1<b_2<b_3<\cdots<b_{k-1}\leqq k+2$$
であり,$b_k=k+3(各桁の和)$である。
$$\therefore \kumiawase{k+2}{k-1}=\kumiawase{k+2}{3}=\bm{\bunsuu16k(k+1)(k+2)(個)}$$
となります。

\end{multicols*}

\newpage

%% 68 %%
\Name{数学IAIIB応用}
\hakosyokika

白い玉が2個,黒い玉が3個,赤い玉が4個ある。これらの玉を次のような条件ですべて使って,一列に並べる。
\begin{enumerate}
\item 玉の色のすべての並べ方は,\Hako 通りである。
\item 白い玉2個が隣り合わない並べ方は,\Hako 通りある。
\item 黒い玉3個が連続している並べ方は,\Hako 通りある。
\item 同じ色な玉は連続しない並べ方は,\Hako 通りある。


\end{enumerate}
%14


\begin{multicols*}{2}
{\bf 【解答】}

\begin{enumerate}
\item $\bunsuu{9\kaizyou}{2\kaizyou3\kaizyou4\kaizyou}=\bm{1260(通り)}$


\item 赤と黒を一列に並べて($\kumiawase73$通り),その間,または両端の8カ所から2カ所選んで白玉を配置すればよい。
$$\therefore \kumiawase73\cdot\kumiawase82=35\cdot28=\bm{980(通り)}$$

\item 白玉○,赤玉◎,黒玉●とする。黒玉3個を一塊と考えて,
\begin{center}
○○\fbox{●●●}◎◎◎◎
\end{center}
を並べればよい。
$$\therefore \bunsuu{7\kaizyou}{2\kaizyou4\kaizyou}=\bm{105(通り)}$$

\item まず,白玉2個,黒玉3個を一列に並べ($\kumiawase52=10$通り),それぞれの並びに対して,赤玉の配置の方法を考える。↑は必ず赤玉を入れるところで,△は入れても入れなくてもよいところとなる。

\begin{enumerate}[m]
\item ${}_{△}○_{↑}○_{△}●_{↑}●_{↑}●_{△}\cdots\kumiawase31=3(通り)$

\item ${}_{△}●_{△}○_{↑}○_{△}●_{↑}●_{△}\cdots\kumiawase42=6(通り)$

\item ${}_{△}●_{↑}●_{△}○_{↑}○_{△}●_{△}\cdots\kumiawase42=6(通り)$


\item ${}_{△}●_{↑}●_{↑}●_{△}○_{↑}○_{△}\cdots\kumiawase31=3(通り)$

\item ${}_{△}○_{△}●_{↑}●_{↑}●_{△}○_{△}\cdots\kumiawase42=6(通り)$

\item ${}_{△}○_{△}●_{△}○_{△}●_{↑}●_{△}\cdots\kumiawase53=10(通り)$


\item ${}_{△}●_{△}○_{△}●_{△}○_{△}●_{△}\cdots\kumiawase64=15(通り)$

\item ${}_{△}●_{↑}●_{△}○_{△}●_{△}○_{△}\cdots\kumiawase53=10(通り)$

\item ${}_{△}●_{△}○_{△}●_{↑}●_{△}○_{△}\cdots\kumiawase53=10(通り)$

\item ${}_{△}○_{△}●_{↑}●_{△}○_{△}●_{△}\cdots\kumiawase53=10(通り)$

\end{enumerate}
\maru1から\maru{10}を合計して,$\bm{79(通り)}$


\end{enumerate}

\columnbreak

{\bf 【コメント】}

『黒玉が3個連続していること』の余事象(否定)は『黒玉が3個連続していないこと』であり,『黒玉が1個も連続していないこと』ではありません。恐らく,集合で数えようとするとかなりの困難を伴うでしょう。

\end{multicols*}
\newpage



%% 69 %%
\Name{数学IAIIB応用}
\hakosyokika
6人がそれぞれプレゼントを持参してパーティーに参加した。参加者が自分以外の誰かにプレゼントを渡すとき,6人全員が1つずつプレゼントを受け取ることができるような渡し方は\karaHako 通りある。
%14

\begin{multicols*}{2}
{\bf 【解答】}

一般に,$n$人に$1$から$n$までの番号を振り,それら$n$個の数字を並べ替えた順列
$$p_1,\,p_2,\,\cdots,\,p_n$$
において,
$$p_1\neqq 1かつp_2\neqq 2 かつ\cdots かつ p_n\neqq n \cdots ☆$$
であるものの個数$f_{n}$がプレゼントの渡し方の総数である。

\begin{itemize}
\item まず,$p_{1}=2$とする。
\item $p_{2}=1$のとき,1と2を入れ換えたときとなる。
$$\begin{hyou}{ccccc}
p_{1}&p_{2}&p_{3}&\cdots\cdots&p_{n}\\
2&1&\multicolumn{3}{c}{(f_{n-2}通り)}\\
\end{hyou}$$

残りの$p_{3}$から$p_{n}$の並べ方は,
\begin{center}
3から$n$までの$n-2$人のプレゼント交換
\end{center}
を考えればよい。この総数は$f_{n-2}$(個)である。



\item $p_{2}\neqq 1$のとき,$p_{2}$から$p_{n}$は$1,\,3,\,4,\,\cdots,\,n$の$n-1$個の数を並べる。
$$\begin{hyou}{cccc}
p_{1}&p_{2}&\cdots\cdots&p_{n}\\
2&\multicolumn{3}{c}{(f_{n-1}通り)}\\
\end{hyou}$$

$p_{2}\neqq 1$なので,$1$を$2$と置き換えて,
\begin{center}
$2,\,3,\,4,\,\cdots,\,n$の$n-1$人のプレゼント交換
\end{center}
を考えればよい。この総数は$f_{n-1}$(個)である。

\item $p_{1}$の置き方は2から$n$までの$n-1$通りあるので,漸化式は$n=3,4\cdots$として,
$$f_{n}=(n-1)\{f_{n-1}+f_{n-2}\}$$
となる。

\end{itemize}
ゆえに,$f_{1}=0$,$f_{2}=1$より,
\begin{align*}
f_4&=(4-1)(2+1)=3\cdot3=9\\
f_5&=(5-1)(9+2)=4\cdot11=44\\
\bm{f_6}&=(6-1)(44+9)=5\cdot53=\bm{265(通り)}
\end{align*}
{\bf 【コメント】}

『完全順列』といいます。高1のときにやったと思うけど,プリントが欲しい人はあげます。
\end{multicols*}
\newpage



%% 70 %%
\Name{数学IAIIB応用}
\hakosyokika
空間に座標系が定められていて,$z$軸上に2点A$(0,\,0,\,6)$,B$(0,\,0,\,20)$が与えられている。$xy$平面上の点P$(x,\,y,\,0)$で,$0\leqq x\leqq 15$,$0\leqq y\leqq 15$,$\kaku{APB}\geqq30\Deg$を満たすものの全体が作る図形の面積を求めよ。
%40


\begin{multicols*}{2}
{\bf 【解答】}
\begin{caprm}

$xy$平面上に$OP=h$となる点Pをとり,\\
$$\theta=\kaku{APB}\geqq30\Deg$$を満たすような$h$の値の範囲を考える。

\begin{center}
\begin{zahyou*}[haiti=t,ul=2mm,yokozikukigou={$x$},tatezikukigou={$z$},gentenhaiti={[sw]},yokozikuhaiti={[s]},tatezikuhaiti={[e]}](-.5,12)(-2,22)
\tenretu{A(0,6)w;B(0,20)w;P(10,0)e;O(0,0)w}
\Arrowline{\O(0,\ymax)}
\Put{(0,\ymax)}{$z$}
\Drawlines{\A\P}
\Takakkei{\O\P\B}
\Tyokkakukigou\P\O\A
\Kakukigou\B\P\A[r](3.5mm,130){$\theta$}
\Kakukigou\A\P\O[r](3.5mm,160){$\alpha$}
\Hen_ko<1>\B\O{20}
\Hen_ko<.2>\A\O{6}
\Hen_ko<.3>\O\P{$h$}

\end{zahyou*}
\end{center}


$$\tan\alpha=\bunsuu6h\cdots\maru1$$
$$\tan(\theta+\alpha)=\bunsuu{20}h\cdots\maru2$$
$$\therefore \bunsuu{\tan\theta+\tan\alpha}{1-\tan\theta\tan\alpha}=\bunsuu{20}{h}$$
$t=\tan\theta$とおいて,\maru1より,
$$\bunsuu{t+\frac6h}{1-t\cdot\frac6h}=\bunsuu{20}{h}$$
$$\therefore \bunsuu{ht+6}{h-6t}=\bunsuu{20}{h}$$
$$\therefore h(ht+6)=20(h-6t)$$
$$\therefore th^2-14h+120t=0$$
$$\therefore t=\bunsuu{14h}{h^2+120}$$
ここで,$t\geqq\tan30\Deg=\bunsuu1{\sqrt 3}$なので,
$$\bunsuu{14h}{h^2+120}\geqq\bunsuu1{\sqrt 3}$$
$$\therefore h^2-14\sqrt 3h+120\leqq 0$$
$$\therefore 7\sqrt 3-3\sqrt 3\leqq h\leqq 7\sqrt 3+3\sqrt 3$$
$$\therefore 4\sqrt 3\leqq h\leqq 10\sqrt 3$$
この$h$の範囲で,Pの存在範囲を考えると,下の図の斜線部分となる。

\begin{center}
\begin{zahyou}[haiti=t,ul=2mm,yokozikukigou={$x$},tatezikukigou={$y$},gentenhaiti={[sw]},yokozikuhaiti={[s]},tatezikuhaiti={[e]}](-2,25)(-2,25)
\tenretu*{A(15,0)w;B(15,15)w;C(0,15);D(20,0);E(20,15);F(0,20);G(15,20)}

\calcval{4*(sqrt(3))}\ra
\calcval{10*(sqrt(3))}\rb
\rtenretu*{P(\rb,30);Q(\rb,60)}



\Put\O{\ougigata**{\rb}{0}{90}}

\Nuritubusi[0]{\B\A\D\E}
\Nuritubusi[0]{\B\C\F\G}
\Put\O{\ougigata*[0]{\ra}{0}{90}}

\Drawlines{\A\B\C}
\Drawlines{\P\O\Q}
\Enko\O{\ra}{0}{90}
\Enko\O{\rb}{0}{90}

\Put\A[s]{15}
\Put\C[w]{15}
\Put{(\rb,0)}[ne]{$10\sqrt 3$}
\Put{(\ra,0)}[s]{$4\sqrt 3$}

\Kakukigou<0>\A\O\P[r](1mm,15){$\bullet$}
\Kakukigou<0>\A\O\P[r](5mm,15){$30\Deg$}

\Kakukigou<0>\P\O\Q[r](1mm,45){$\bullet$}
\Kakukigou<0>\Q\O\C[r](1mm,65){$\bullet$}
\end{zahyou}
\end{center}


\end{caprm}
求める面積$S$は
\begin{align*}
S&=(10\sqrt3)^2\pi\cdot\bunsuu{30}{360}+15\cdot5\sqrt 3-(4\sqrt3)^2\pi\cdot\bunsuu{1}{4}\\
&=25\pi+75\sqrt 3-12\pi\\
&=\bm{13\pi+75\sqrt 3}
\end{align*}

{\bf 【コメント】}

これは昔の東大の問題なのですが,見通しよくやらないと泥沼です。計算量もあります。角度の条件は,ベクトル($\cos$)か,直線の傾き($\tan$)で処理します。
\end{multicols*}
\newpage


\end{document}
